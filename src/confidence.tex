\chapter{Confidence}

Research on \gls{confidence} spreads across several subfields of psychology and neuroscience.

...

\section{Terminology}

Everyone knows intuitively what \gls{confidence} is about. For example, before taking an exam, a student feels more or less confident regarding his/her chances of success. However, coming up with a clear definition of the concept of confidence is not straightforward. Several related concepts also have to be specified along the way.

\Gls{confidence} has been defined as a belief about the validity of our own thoughts, knowledge or performance that relies on a subjective feeling \cite{grimaldiThereAreThings2015}. This psychological, human-centered definition has been challenged by recent advances in a nascent field: the neuroscience of \gls{confidence}. In particular, studies have shown that the sense of \gls{confidence}, from both behavioral and neuronal standpoints, is not specifically humane but shared with other mammals like monkeys or rodents \cite{kepecsNeuralCorrelatesComputation2008}. Moreover, concepts like "belief", "thought" and "feeling" are problematic from a neuroscientific perspective. In order to reconcile the psychology and neuroscience approaches of \gls{confidence}, a more generic definition is needed.

A general understanding of the notion of \gls{confidence} is that it fundamentally quantifies a degree of belief, or synonymously, a degree of reliability, trustworthiness, certitude, or plausibility \cite{meynielConfidenceBayesianProbability2015}. More formally, \gls{confidence} can be defined as a Bayesian probability: a subjective expectation about something or someone. This implies that \gls{confidence} is a form of certainty \cite{pougetConfidenceCertaintyDistinct2016}, which calls for a clarification of the relationship between \gls{confidence} and (un)certainty. At first glance, the notions of "\gls{confidence}" and "\gls{uncertainty}" merely seem like the inverse (or reciprocal) of one another \cite{meynielConfidenceBayesianProbability2015}. However, a distinction can be made.

Generally speaking, \gls{uncertainty} (or incertitude) characterizes situations involving imperfect, noisy or unknown information. To perform well, the brain needs to be effective at dealing with many uncertainties, some of them external (changes in world state or sensorimotor variability), others internal (cognitive variables, timing or abstract states). For example, driving one's car requires processing different sensory inputs (each corrupted by noise) in order to adapt to a continuous stream of not-so-predictable external events (a child or an animal crossing the street, the car in front breaking suddenly, and so on). \cite{pougetConfidenceCertaintyDistinct2016}. \Gls{uncertainty} is inherent to all stages of neural computation and optimal behavior requires adapting to such uncertainty \cite{flemingMetacognitionConfidenceReview}. \Gls{uncertainty} depends upon the noise properties of the considered phenomenom, including both the degree of stochasticity in its measurement (expected uncertainty or observation noise, the variance of which is called \textit{stochasticity}) and how quickly or how often its change (unexpected uncertainty or process noise, the variance of which is known as \textit{volatility}) \cite{pirayModelLearningBased2021}.

From these premises, disentangling confidence from uncertainty become easier. Uncertainty refers to probabilistic representations of information, and confidence refers to scalar values (for example, a numerical rating on a scale) derived from those distributions. The conceptual relationship between confidence and uncertainty is very much the same as the relationship between "summary statistics" (mean, standard deviation, etc.) and the data they describe \cite{meynielConfidenceBayesianProbability2015}. This distinction is supported by research about the neural basis of confidence \cite{pougetConfidenceCertaintyDistinct2016}, \cite{flemingMetacognitionConfidenceReview}.
