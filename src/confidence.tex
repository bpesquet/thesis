\chapter{Confidence}

Research on \gls{confidence} spreads across several subfields of psychology and neuroscience.

...

\section{Terminology}

Everyone knows intuitively what \gls{confidence} is about, yet it is seldom defined explicitely. Thus, it is essential to start by precisely defining \gls{confidence} and the different concepts related to it.

\Gls{confidence} has been defined as a belief about the validity of our own thoughts, knowledge or performance that relies on a subjective feeling \cite{grimaldiThereAreThings2015}. This psychological, human-centered definition has been challenged by recent advances in a nascent field: the neuroscience of \gls{confidence} \todo{Citation neurosciance of \gls{confidence}}. In particular, studies  has shown that the sense of \gls{confidence}, from both behavioral and neuronal standpoints, is not specifically humane but shared with other mammals like monkeys or rodents \cite{kepecsNeuralCorrelatesComputation2008}. Moreover, concepts like "belief", "thought" and "feeling" are problematic from a neuroscientific perspective. In order to reconcile the psychology and neuroscience approaches of \gls{confidence}, a more generic definition is needed.

A general understanding of the notion of \gls{confidence} is that it fundamentally quantifies a degree of belief, or synonymously, a degree of reliability, trustworthiness, certitude, or plausibility \cite{meynielConfidenceBayesianProbability2015}. More formally, \gls{confidence} can be defined as a Bayesian probability: a subjective expectation about something or someone. This implies that \gls{confidence} is a form of certainty, which calls for a clarification of the relationship between \gls{confidence} and (un)certainty.

Generally speaking, \gls{uncertainty} (or incertitude) characterizes situations involving imperfect, noisy or unknown information. To perform well, the brain needs to be effective at dealing with many uncertainties, some of them external (changes in world state or sensorimotor variability), others internal (cognitive variables, timing or abstract states) \cite{pougetConfidenceCertaintyDistinct2016}. Uncertainty is inherent to all stages of neural computation and optimal behavior requires sensitivity to such uncertainty \cite{flemingMetacognitionConfidenceReview}. \Gls{uncertainty} depends upon the noise properties of the considered information, including both the degree of stochasticity in its measurement (expected uncertainty or observation noise, the variance of which is called \textit{stochasticity}) and how quickly or how often its change (unexpected uncertainty or process noise, the variance of which is known as \textit{volatility}) \cite{pirayModelLearningBased2021}.

At first glance, the notions of "\gls{confidence}" and "\gls{uncertainty}" merely seem like the inverse (or reciprocal) of one another \cite{meynielConfidenceBayesianProbability2015}. However, several key distinctions can be made.

\section{Properties}

\section{Reporting}